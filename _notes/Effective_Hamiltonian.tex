\documentclass[
    % reprint,
    preprint,
    amsmath,amssymb,
    aps,
    prb,
    floatfix,
]{revtex4-2}

\usepackage{graphicx}
\usepackage{dcolumn}
\usepackage{bm}
\usepackage{lipsum}
\usepackage{framed}
\usepackage{braket}
%\usepackage{hyperref}

\begin{document}

\title{Flip-Flop Qubit Coupling Mediated by a Microwave Cavity: Effective Hamiltonian}

\author{John Truong}
\author{Xuedong Hu}%
 \email{xhu@buffalo.edu}
\affiliation{%
 University at Buffalo\\
 jtruong@buffalo.edu
}

\date{\today}

\begin{abstract}
    
\end{abstract}

\maketitle

\section{Full Hamiltonian}
    We first start by providing the full Hamiltonian for a multiqubit system with a microwave cavity.
    \begin{equation}
        H = H_q + H_{cav} + H_{int} \,.
    \end{equation}
    The individual parts of this Hamiltonian are:
    \begin{align}
        H_q &= -\frac{\hbar}{2}\sum_{i} \omega_{\sigma,i} \sigma_{z,i} + \omega_{\tau,i} \tau_{z,i} \\
        H_{cav} &= \hbar\omega_{cav}a^\dagger a \\
        H_{int} &= \frac{\hbar}{2}\sum_i g(a^\dagger+a)(1+Z_i) \,,
    \end{align}
    where the summation over $i$ is over all donors. The $\omega_{\sigma,i}$ and $\omega_{\tau,i}$ are the charge and flip-flop qubit energies for each donor, respectively, while $\omega_{cav}$ is the cavity energy. $g$ is the coupling strength of the cavity to the electric dipole. $a$ and $a^\dagger$ are the usual photonic annihilation and creation operators. Lastly, the $(1+Z_i)/2$ terms are the electric dipole moments of each donor. $Z_i$ can also be written in terms of the Pauli operators,
    \begin{equation}
        Z_i = \sum_{jk}z_{jk,i}\sigma_{j,i}\tau_{k,i} \,.
    \end{equation}

\section{Rotating Wave Approximation}
    In order to better work with the above Hamiltonian, we perform the rotating wave approximation (RWA). In order to do this, we operate under a set of assumptions:
    \begin{framed}
        \begin{enumerate}
            \item The flip-flop qubit for the two lowest energy states for each donor: $\omega_\tau < \omega_\sigma$
            \item The flip-flop and charge qubits and the cavity all have energies of the same order of magnitude: $\omega_{cav}\approx\omega_\sigma\approx\omega_\tau$
        \end{enumerate}
    \end{framed}

    To perform the RWA, the steps are outlined below:
    \begin{enumerate}
        \item Transform into the interaction picture.
        \item Drop the counter-rotating terms (i.e. the fast oscillating terms).
        \item Transform back into the Schroedinger picture.
    \end{enumerate}

    \subsection{Interaction Picture}
    An operator expressed in the interaction picture is given by:
    \begin{equation}
        A(t) = e^{iH_0t/\hbar}Ae^{-iH_0t/\hbar} \,.
    \end{equation}
    which is the solution to the differential equation
    \begin{equation}
        i\hbar\frac{\partial}{\partial t}A(t) = \left[A,H_0\right]
    \end{equation} 
    Here, we will use the explicit time dependence to indicate an operator in the interaction picture. The $H_0$ is the non-interactive part of the Hamiltonian. In our case, $H_0 = H_{cav} + H_q$. Ultimately, our goal is to obtain $H_{int}(t)$. To do this, we first obtain each of the operators present in $H_{int}$ in the interaction picture individually.

    a) \emph{The cavity operators, $a$ and $a^\dagger$}.
    \begin{align*}
        \dot{a}(t) &= -\frac{i}{\hbar}\left[a,H_0\right] = -\frac{i}{\hbar}\left[a,\hbar\omega_{cav}a^\dagger a\right] = -i\omega_{cav}a \\
        a(t) &= ae^{-i\omega_{cav}t} \\
        a^\dagger(t) &= a^\dagger e^{i\omega_{cav}t}
    \end{align*}

    b) \emph{The Pauli z operators, $\sigma_z$ and $\tau_z$}.
    \begin{align*}
        \dot{\sigma}_z(t) &= -\frac{i}{\hbar}\left[\sigma_z,H_0\right] = -\frac{i}{\hbar}\left[\sigma_z,-\frac{\hbar}{2}\omega_\sigma\sigma_z\right] = 0 \\
              \sigma_z(t) &= \sigma_z \\
                \tau_z(t) &= \tau_z
    \end{align*}

    c) \emph{The Pauli x and y operators, $\sigma_{x(y)}$ and $\tau_{x(y)}$}.
    For this set of operators, it is easier to instead work with the raising and lowering operators $\sigma_\pm (\tau_\pm)$. We'll define the charge operators as:
    \begin{equation}
        \sigma_\pm = \sigma_x \mp i\sigma_y
    \end{equation}
    and similarly for the flip-flop ones.
    \begin{framed}
        Note: This definition is opposite from the standard convention of $\sigma_\pm = \sigma_x \pm i\sigma_y$. This is because of the ordering of the energy eigenvalues. The usual $\sigma_+$ will raise the eigenvalue of the $\sigma_z$ operator to the next value. However since the eigenvalues of $\sigma_z$ are ordered from high to low while our energies are ordered from low to high, the standard $\sigma_+$ would lower our energy state. The flipped definition will keep it clear that the raising operator increases the energy.
    \end{framed}
    The inverse equations are then:
    \begin{align}
        \sigma_x &= \sigma_+ + \sigma_- \\
        \sigma_y &= i\sigma_+ - i\sigma_-
    \end{align}
    We now get the raising and lowering operators in the interaction picture.
    \begin{align*}
        \dot{\sigma}_\pm(t) &= -\frac{i}{\hbar}\left[\sigma_\pm,H_0\right] \\
                            &=-\frac{i}{\hbar}\left[\sigma_\pm,-\frac{\hbar}{2}\omega_\sigma\sigma_z \right] \\
                            &= \frac{1}{2}i\omega_\sigma \left[\sigma_x\mp i\sigma_y,\sigma_z \right] \\
                            &= \frac{1}{2}i\omega_\sigma\left( -2i\sigma_y \pm 2\sigma_x \right) \\
                            &= i\omega_\sigma\left( -i\sigma_y \pm \sigma_x \right) \\
                            &= \pm i\omega_\sigma(\sigma_x\mp i\sigma_y) \\
                            &= \pm i\omega_\sigma\sigma_\pm\\
                \sigma_\pm(t) &= \sigma_\pm e^{\pm i \omega_\sigma t}
    \end{align*}
    Now we can write the Pauli $x$ and $y$ matrices in the interaction picture.
    \begin{align*}
        \sigma_x(t) &= \sigma_+e^{i\omega_\sigma t} + \sigma_-e^{-i\omega_\sigma t} = \sigma_x(e^{i\omega_\sigma t}+e^{-i\omega_\sigma t})-i\sigma_y(e^{i\omega_\sigma t}-e^{-i\omega_\sigma t}) \\
        \sigma_y(t) &= i\sigma_+e^{i\omega_\sigma t} -i \sigma_-e^{-i\omega_\sigma t} = \sigma_y(e^{i\omega_\sigma t}+e^{-i\omega_\sigma t})+i\sigma_x(e^{i\omega_\sigma t}-e^{-i\omega_\sigma t})  \\
        \tau_x(t) &= \tau_+e^{i\omega_\tau t} + \tau_-e^{-i\omega_\tau t} = \tau_x(e^{i\omega_\tau t}+e^{-i\omega_\tau t})-i\tau_y(e^{i\omega_\tau t}-e^{-i\omega_\tau t})\\
        \tau_y(t) &= i\tau_+e^{i\omega_\tau t} -i \tau_-e^{-i\omega_\tau t}= \tau_y(e^{i\omega_\tau t}+e^{-i\omega_\tau t})+i\tau_x(e^{i\omega_\tau t}-e^{-i\omega_\tau t})\\ \\
    \end{align*}



    \subsection{Dropping Terms}
    The interaction Hamiltonian is:
    \begin{equation}
        H_{int}(t) = \frac{\hbar g}{2}\sum_i (a^\dagger(t)+a(t))+\frac{\hbar g}{2}\sum_i\sum_{jk}z_{jk,i}(a^\dagger(t)+a(t))\sigma_{j,i}(t)\tau_{k,i}(t) \,.
        \label{Hint_I}
    \end{equation}

    The first term in the above equation accounts for vacuum fluctutations. We will neglect these terms for now. Furthermore, terms in the second summation that only include Pauli $z$ operators and the identity also only contribute to vacuum fluctutations and will also be ignored.

    Of the remaining terms, we only keep ones where there is potential for energy conservation. Since $a(t)$ would absorb a photon, we only keep terms that increase the qubit energy state. Conversely, for $a^\dagger(t)$, we keep only terms that decrease the energy state.

    What's left is
    \begin{multline}
        H_{int}(t) = \frac{\hbar g}{2}\sum_i a(t)\left[ (z_{01}+iz_{02})\tau_+(t) + (z_{10}+iz_{20})\sigma_+(t) \right. \\ 
        + (z_{11} + iz_{21})\sigma_+(t)\tau_x(t) +(z_{12}+iz_{22})\sigma_+(t)\tau_y(t) \\
        \left. +(z_{13}+iz_{23})\sigma_+(t)\tau_z(t) + (z_{31}+iz_{32})\sigma_z(t)\tau_+(t) \right] + h.c.
    \end{multline}

    While at this point, we are still entirely general, we can apply conditions specific to the flip-flop system to simplify the equation further. First we eliminate the $z_{jk}$ that are exactly equal to zero.
    \begin{multline}
        H_{int}(t) = \frac{\hbar g}{2}\sum_i a(t)\left[ (z_{01}+z_{31}\sigma_z(t))\tau_+(t) + (z_{10}+z_{13}\tau_z(t))\sigma_+(t) \right. \\
        \left.  +( z_{11}-z_{22})\sigma_+(t)\tau_+(t)+(z_{11}+z_{22})\sigma_+(t)\tau_-(t)\right] + h.c.
    \end{multline}
    Next, $z_{01} \ll z_{31}$ and $z_{10} \gg z_{13}$ so we can neglect those terms. We define new constants
    \begin{align}
        g_\sigma &= \frac{1}{2}gz_{10} \\
        g_\tau &= \frac{1}{2}gz_{31} \\
        g_\pm &= \frac{1}{2}g(z_{11}\pm z_{22}) 
    \end{align}
    so that the final Hamiltonian in the interaction picture is
    \begin{equation}
        H_{int}(t) = \hbar\sum_ia(t)\left[ g_{\sigma,i}\sigma_{+,i}(t)+g_{\tau,i}\sigma_{z,i}\tau_{+,i}(t) + g_{+,i}\sigma_{+,i}(t)\tau_{-,i}(t) + g_{-,i}\sigma_{+,i}(t)\tau_{+,i}(t) \right] + h.c.
    \end{equation}
    
    \subsection{Back to Schroedinger Picture}
    Our effective Hamiltonian is finally
    \begin{equation}
        H_{int} = \hbar\sum_ia\left[ g_{\sigma}^{(i)}\sigma_{+}^{(i)}+g_{\tau}^{(i)}\sigma_{z}^{(i)}\tau_{+}^{(i)} + g_{+}^{(i)}\sigma_{+}^{(i)}\tau_{-}^{(i)} + g_{-}^{(i)}\sigma_{+}^{(i)}\tau_{+}^{(i)} \right] + h.c.
    \end{equation}
    To simplify analysis, we split this into two parts $H_{int} = H_1 + H_2$ where
    \begin{align}
        H_1 &= \hbar\sum_ia\left[ g_{\sigma}^{(i)}\sigma_{+}^{(i)}+g_{\tau}^{(i)}\sigma_{z}^{(i)}\tau_{+}^{(i)} \right] + h.c. \\
        H_2 &= \hbar\sum_ia\left[ g_{+}^{(i)}\sigma_{+}^{(i)}\tau_{-}^{(i)} + g_{-}^{(i)}\sigma_{+}^{(i)}\tau_{+}^{(i)} \right] + h.c. \,.
    \end{align}
    $H_1$ conserves the total number of excitations (photons + flip-flop + charge), $N$, while $H_2$ does not. This does mean that $H_0 + H_1$ is block diagonal with respect to $N$. This is conducive to applying a Schrieffer-Wolff transformation to fold $H_2$ into those blocks, turning an infinite Hilbert space to an infinite set of finite ones.

    \subsection{Projecting Onto $N$ basis}
    To better perform this, we should first project each of the Hamiltonians into a basis descibed by the total number of excitations, $N$, rather the photon occupancy $n$. First, let's clarify some notation. When in the photon basis, states will will expressed like so: $\ket{\psi} = \ket{n}\ket{J}=\ket{n,J}$, where $\ket{n}$ is the cavity photon occupation and $\ket{J}=\Pi_i\ket{J_i}$ forms the set of qubit states. In the total excitation basis, we'll use $\ket{\psi} = \ket{N}'\ket{J}=\ket{N,J}'$ where $\ket{N}'$ is the total number of excitations.

    We also define a counting operator $P_i$ that counts the number of excitation on a qubit.
    \begin{equation}
        P_i\ket{J_i} = \textrm{diag}(0,1,1,2) = 1-\frac{1}{2}\sigma_z^{(i)}-\frac{1}{2}\tau_z^{(i)}
    \end{equation}
    With this, we can partially project our Hamiltonian using a new operator $N = a^\dagger a + \sum_i P_i$.
    \begin{equation}
        H_0 = \hbar\omega_{cav}N - \frac{1}{2}\sum_i \left(\delta_\sigma^{(i)}\sigma_z^{(i)}+\delta_\tau^{(i)}\tau_z^{(i)}\right) \,
    \end{equation}
    where $\delta_{\sigma(\tau)} = \omega_{\sigma(\tau)}-\omega_{cav}$.
    
    Unfortunately, there is no simple and general way to project $a$ and $a^\dagger$ onto the $N$ basis for an arbitrary number of qubit systems so $H_1$ and $H_2$ will have to be dealt with on a case-by-case basis for varying values of the total number of donors $Q$. Since we are interested in two-qubit coupling first and foremost, the rest of the work will be on one and two-qubit systems.
    
    \section{Single Qubit System}
    \subsection{Matrix Representation}
        We can first start by expressing $H$ in matrix form.
        \begin{align}
            \bra{N}H_0\ket{N}&=H^0_N= \hbar\omega_{cav}NI_4 + \frac{\hbar}{2} \begin{pmatrix}
                 - \delta_\sigma - \delta_\tau & 0 & 0 & 0 \\
                0& - \delta_\sigma + \delta_\tau & 0 & 0 \\
                0&0&+ \delta_\sigma - \delta_\tau  & 0 \\
                0 & 0 & 0 & + \delta_\sigma + \delta_\tau
            \end{pmatrix} \\
            &= \hbar\omega_{cav}NI_4-\frac{\hbar}{2}(\delta_\sigma\sigma_z+\delta_\tau\tau_z) \\
            \bra{N}H_1\ket{N}&=H^1_N=\begin{pmatrix}
               0 & \sqrt{N}g_\tau & \sqrt{N}g_\sigma & 0 \\
               \sqrt{N}g_\tau & 0 & 0 & \sqrt{N-1}g_\sigma \\
               \sqrt{N}g_\sigma & 0 & 0 & -\sqrt{N-1}g_\tau \\
               0 & \sqrt{N-1}g_\sigma & -\sqrt{N-1}g_\tau & 0
           \end{pmatrix} \\
           &=\frac{\hbar}{2}g_\sigma\left(\sqrt{N}\sigma_x(1+\tau_z)+\sqrt{N-1}\sigma_x(1-\tau_z)\right)+\frac{\hbar}{2}g_\tau\left(\sqrt{N}(1+\sigma_z)\tau_x-\sqrt{N-1}(1-\sigma_z)\tau_x\right)\\
            \bra{N+1}H_2\ket{N}&=H^2_N= \sqrt{N}\hbar\begin{pmatrix}
                0 & 0 & 0 & 0 \\
                0 & 0 &  g_+ & 0 \\
                0 & 0 & 0 & 0 \\
                 g_- & 0 & 0 & 0
            \end{pmatrix} \\
            &= \sqrt{N}\hbar(g_+\sigma_-\tau_++g_-\sigma_+\tau_+) \\
            \bra{N}H_2\ket{N+1} &= H_N^{2\dagger}
        \end{align}
    \subsection{Schrieffer-Wolff Transform}
        \begin{framed}
            From the Winkler textbook:
            \begin{equation}
                \tilde{H} = e^{-S}He^{S}
            \end{equation}
            where $S$ is the transformation matrix and can be expanded in orders $S = S^{(1)} + S^{(2)} + \dots$.
            \begin{align}
                S^{(1)}_{ml} &= -\frac{H'_{ml}}{E_m-E_l}\\
                S^{(2)}_{ml} &= \frac{1}{E_m-E_l}\left[ \sum_{m'}\frac{H_{mm'}H_{m'l}}{E_{m'}-E_l}-\sum_{l'}\frac{H_{ml'}H_{l'l}}{E_m-E_{l'}} \right]
            \end{align}
        \end{framed}
        From the formulas given by Winkler,
        \begin{align}
            \bra{N+1}S^{(1)}\ket{N} &= S_N^{(1)} = \begin{pmatrix}
                0 & 0 & 0 & 0 \\
                0 & 0 & -\frac{\sqrt{N}g_+}{\omega_{cav}-\delta_\sigma+\delta_\tau} & 0 \\
                0 & 0 & 0 & 0 \\
                -\frac{\sqrt{N}g_-}{\omega_{cav}+\delta_\sigma+\delta_\tau} & 0 & 0 & 0
            \end{pmatrix}
        \end{align}
        However, this restricts our analysis to parameters so that the cavity is not in resonance with the $0\leftrightarrow3$ and $1\leftrightarrow2$ transitions.
    \subsection{Effective Hamiltonian}
        The overall effective Hamiltonian is then:
        \begin{align}
            H^{(0)} &= \hbar\omega_{cav}NI_4-\frac{\hbar}{2}(\delta_\sigma\sigma_z \\
            H^{(1)} &= \frac{\hbar}{2}g_\sigma\left(\sqrt{N}\sigma_x(1+\tau_z)+\sqrt{N-1}\sigma_x(1-\tau_z)\right)+\frac{\hbar}{2}g_\tau\left(\sqrt{N}(1+\sigma_z)\tau_x-\sqrt{N-1}(1-\sigma_z)\tau_x\right) \\
            H^{(2)} &= \frac{\hbar}{4}\left[ -\frac{Ng_-^2}{\omega_{cav}+\delta_\sigma+\delta_\tau}(1+\sigma_z)(1+\tau_z) +\frac{(N-1)g_+^2}{\omega_{cav}-\delta_\sigma+\delta_\tau}(1+\sigma_z)(1-\tau_z) \right.\nonumber\\
            &\qquad\qquad\left. -\frac{Ng_+^2}{\omega_{cav}-\delta_\sigma+\delta_\tau}(1+\sigma_z)(1-\tau_z)+\frac{(N-1)g_-^2}{\omega_{cav}+\delta_\sigma+\delta_\tau}(1-\sigma_z)(1-\tau_z) \right]
        \end{align}
        Notice that the second order part simply accounts for small energy corrections.
    \section{Two Qubit System}
    \subsection{Effective Hamiltonian}
    We apply a similar technique to the single qubit case so those steps are omitted. In addition, apply a second SW transformation to reduce our Hamiltonian to the following states: $\ket{00}$, $\ket{01}$, $\ket{10}$, and $\ket{11}$. Lastly, we make the assumption that the two qubits are under symmetric parameters.
    The effective Hamiltonian is then:
    \begin{align}
        H_0 &= \hbar(\omega_{cav}N-\delta_\sigma-\frac{2g_\sigma^2}{\delta_\sigma}(N-1))+\hbar(-\delta_\tau-\frac{2g_\sigma^2}{\delta_\sigma})\cdot\textrm{diag}\left\{1,0,0,-1\right\} \\
        V_1 &= \hbar\begin{pmatrix}
            0 & \sqrt{N}g_\tau & \sqrt{N}g_\tau & 0 \\
            \sqrt{N}g_\tau &0 & 0 & \sqrt{N-1}g_\tau \\
            \sqrt{N}g_\tau & 0 & 0 & \sqrt{N-1}g_\tau \\
            0              & \sqrt{N-1}g_\tau & \sqrt{N-1}g_\tau & 0
        \end{pmatrix} \\
        V_2 &= \hbar\cdot\textrm{diag}\left\{ -N\frac{2g_-^2}{\omega_{cav}+\delta_\sigma+\delta_\tau},(N-1)(\frac{g_+^2}{\omega_{cav}-\delta_\sigma+\delta_\tau}-\frac{g_-^2}{\omega_{cav}+\delta_\sigma+\delta_\tau}), \right.\nonumber\\
        &\qquad\qquad\qquad \left. (N-1)(\frac{g_+^2}{\omega_{cav}-\delta_\sigma+\delta_\tau}-\frac{g_-^2}{\omega_{cav}+\delta_\sigma+\delta_\tau}),(N-2)\frac{2g_+^2}{\omega_{cav}-\delta_\sigma+\delta_\tau} \right\}
    \end{align}
    Notice again that the second order part simpy accounts for energy corrections.
\end{document}